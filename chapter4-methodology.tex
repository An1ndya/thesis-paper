\chapter{Methodology}\label{ch4}

\section{Proposed Methodology}
In order to build a knowledge graph from tweets, we can divide our problem into these sub-problems. They are:
\begin{itemize}
\item Extracting information / tuples from tweets 
\item Merging those tuples with External Knowledge base
\item disambiguating after Merging 
\end{itemize} 
These two hypothesis are elaborated below: 

\Section{ Extracting Information or Tuples from tweets }
We know that tweets have very limited data. for example, 
\textit{ @ImRo45 is back in the Test squad for India’s tour of Australia. } is a tweet. Someone with adequate knowledge in cricket
may assume that this is about a game but otherwise, there is very little information in this tweet. 
Also, words like prepositions (of, in, for) , articles (a, the) have almost no value. On the contrary, `India', `tour', `Australia' 
are inpoetant words here. 
So, we have to extract these \textit{ important } words into \textit{ tuples } of the form $<u, e, v>$ . \\
 Define tuple of a graph $<u, e, v>$ where $ u $ is starting node, $ v $ is ending node and $ e $ is an edge between $ u $ and $v $. \\

this can be done in two ways: 
\begin{itemize}
\item Machine Learning 
\item JAPE 
\end{itemize} 
\\
In machine learning approach, one has to divide the dataset into `training dataset'  and `test dataset' . 
After that, one has to manually annotate the test dataset by marking possible tuples. After that, some machine learning algorithm
 can be trained on the test dataset and finally use it on the test dataset. 
\\ 
JAPE stands for `Jolly And Pleasant Experience'. It is a special kind of pattern matching language in GATE developer. 
By writting rules in JAPE , one can fine-tune GATE developer to extract necessary information which is in this case, 
finding the appropriate tuples. 
\\ 
In our methodology, we took the liberty to use JAPE. The reason is that, in previous works, other researchers  
 have already used machine learning approaches in this stage. Besides, we know 
that a machine learning algorithm is only as good as its training dataset. Using JAPE, detect tuples 
better. How we can use JAPE is expained below: 

\textbf{ CASE 1: } \\
First we start by writing rules for simple sentences. For example, \textit{ CR7 has joined Juventus }. Our target is to 
convert this sentences into tuple form like $<u,e,v> = <CR7 , join, juventus > $. So our initial target is to
 write JAPE rules for these kinds of sentences or tweets. \\ 

\textbf{ CASE 2: } \\
Practically, sentences like case 1 donot provide much information. So now we need to extract 
information from the surrounding context. For example, \textit{ segment tree is a good data structure. } 
The previous rule may create tuples like <tree, is data>. This is ambiguous as we cannot differentiate  But in this case we want <segment tree, is, data structure>. 
So we have to extract words like `segment' that will allow us to disambiguate between `plants' and `data structure trees'.

\textbf{ CASE 3: } \\ 
Though tweets are very small, we can occationally come across tweets with multiple (two or three) sentences. In that case,
we have to Extract informations from multiple sentences.




\section{ Merging Tuples with External Knowledge Base }
Since we have very limited data in our tweets, we need the help of an external knowledge graph to build 
up our version of knowledge graph. Suppose, in the previous step, we got this tuple: \testit{Alice loves Bob} here $<u, e, v> = <Alice, love, Bob>$.
Now the problem is that we need to identify who Alice and Bob really are, ie, if Alice is an actor , a fictional character or someone else.

For this can use Freebase, DBpedia etc as a source of external knowledge graph.
In our case, we took the liberty to use DBpedia. How to extract data from DBpedia is given below: 
\subsection{SPARQL}
One way to get data from DBpedia is to use the SPARQL. We can write queries in this language, such as 

In our case , we have nodes $ u $ and $ v $, so we can fetch datas that are related to both of them and make an intersection of those. 
But practically, this intersection returns a huge list of tentative tuples.  We can set a limit in our query to 
restrict our results' size. But still we need to disambiguate the results. For this, the next section comes into play.

\section{ disambiguating after Merging }
For each tuple <u,e,v> , we have a set of tentative candidates $<a_1, b_1>$, $<a_2, b_2>$, $<a_n, b_n>$ from dbpedia. 
We have to select only one of them. For this there are several ways:
\subsection{Calculating Relatedness}
 We can compute $ relatedness $ of two articles , $a$ , $b$ using the formula \\
  $ relatedness(a,b) = \frac{\log(max(|A|, |B|)) - \log(A \cap B) }{ \log( |W| ) - \log(max(|A|, |B|)) } $ \\

where $a$ , $b$ are the articles of our interest, $A$  = set of all articles linking to $a$, $B$ = set of all articles linking to $b$, 
and $W$ = set of all links in DBpedia. The formula generates a score for each candidate tuples and we select te one with the highest score.
This is the easiest way to perform disambiguation.   

\subsection{ Word HAshing with letter tri-gram }
In this way, we can we take a candidate $<a_i, b_i>$, then we read the articles by converting them into $n-grams$, 
then we calculate the angle between them. The closer the articles are, the better result we can get. 

\subsection{ Deep Neural Network } 
One can also use a dump from DBpedia, and follow the detailed instructions provided here ~\cite{ref1DeepLearning} 
to create a deep neural network that can be used then to link tweet tuples with a corresponding article.

\endinput